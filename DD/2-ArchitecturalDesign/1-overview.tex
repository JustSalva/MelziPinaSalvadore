In this section we will present a general overview of Travlendar+, with a specific focus on the main logical components and their interactions. \\ \\
The main high-level components of the system-to-be are:
\subsubsection{Mobile App:}
\label{subsubsect:Mobile App}
The Mobile App is a presentation layer that lets the users access the functionalities offered by the system-to-be on their smartphones and tablets. The mobile application offers also an internal logic that handles:
\begin{itemize}
	\item the notifications reception, through the services offered by Google Cloud Messaging APIs;
	\item the map and path drawing functionalities offered through the interaction with Google Maps mobile APIs.
\end{itemize}  

\subsubsection{Web Browser:}
\label{subsubsect:Web Browser}
The web browser is a second presentation layer that lets the users access the functionalities offered by Travlendar+ on their browsers.
It relies on the connection with the web server in order to obtain dynamic web pages.

\subsubsection{Web Server:}
\label{subsubsect:Web Server}
This layer provides dynamic web pages for the web-based application, it communicates directly with the application server to satisfy the client requests, using proper interfaces. This layer interacts also with an external system (Google Maps API) in order to display an embedded map containing the user travel's information.

\subsubsection{Application Server:}
\label{subsubsect:Application Server}
This is the logic layer that implements all the core functionalities of the system. It receives and replies to client's requests, if required it sends notifications to the mobile applications, it interacts with external systems in order to satisfy the user's requests and it interacts with a DBMS in order to guarantee the information's persistence.\\
In particular, the application server will interact with these external systems:
\begin{itemize}
\item Google Maps APIs, in order to provide feasible paths according to the user preferences;
\item Transport Service provider's systems, in order to offer functionalities that allow the user to arrange his trips, such as ticket buying, location of sharing vehicles and strikes information;
\item Weather APIs, in order to retrieve weather information and to apply possible travel constraints related to the weather;
\item Google Cloud Messaging APIs, in order to send notifications to the user's mobile apps.
\end{itemize} 

\subsubsection{Database Server:}
\label{subsubsect:Database Server}
The data layer, that supports all data storage and management operations. This layer ensures that ACID properties are satisfied. The database server will interact only with the application server, so that the data will never be exposed directly to the client layers.

\subsubsection{External Systems:}
\label{subsubsect:External Systems}
Those systems are not internal components of our application, but the system-to-be will have to interact with them in order to guarantee all the system functionalities. \\
Most of their interactions with the system have already been described in the previous paragraphs, but here we will explain in detail the interaction with Transport Service providers: Travlendar+ will initially integrate some external transport services systems (public transport service providers and sharing vehicles providers) but will also offer proper APIs to allow other Transport Service providers to interact with Travlendar+ and therefore to be considered as suggested travel means to the users.

\begin{figure}[H]
\begin{center}
		\hspace*{-50pt}
		\includegraphics[scale=0.2]{GeneralArchitecture.png}
\end{center}
\caption{General Architecture}
\end{figure}
