\subsection{Authentication}
When the users log into the system for the first time from a device, the Application server will assign to that device an univocal Unicode code, so for further requests from that device (web browser or mobile application) the users can avoid logging in again. Every new request will include into the request payload that code, enabling the Application server to validate the request and to recognize the specific user that sent them. The univocal code will be generated by an internal algorithm in the AuthenticationManager (explained in section \ref{sect: Univocal code}). The AuthenticationManager will check every time the consistency of the univocal code and it will also check if it comes from the same device it is associated to.
If a login is performed from a device already registered, but it comes from a different Travlendar+ account, the previous univocal code associated to that device will be deleted, and the previous user will have to log in again with his previous account.

\subsection{Notifications}
In order to send notifications to mobile devices our system will memorize the Identifiers of the user devices, and when a notification is to be sent the Application Server will interact with the interfaces offered by GCM (Google Cloud Messaging) in order to actually forward them to the user devices. We choose to use GCM services cause they handle all aspects of the queueing of messages and delivery to client applications running on different target devices, and because it is completely free.

\subsection{Periodic events}
Once a day, an internal module is run to check that every periodical event inserted by a user is covered for a year, e.g. if an event like "Friday lunch with parents" with a weekly periodicity is created on 2018/01/05, on 2018/01/12 the module will create a new event on 2019/01/12.