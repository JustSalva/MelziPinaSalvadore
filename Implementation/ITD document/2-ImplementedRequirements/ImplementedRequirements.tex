Our first release of Travlendar+ includes all core functionalities of the application server and of the database server, as specified in the previous documents (more details in section \ref{sec:ApplAndDBServers}), a set of initial functionalities in the Android application (more details in section \ref{sec:AndroidApp}) and does not include either the web server implementation nor the IOs app.
In the following schema we've highlighted the components of our general architecture (see also section 2.1 of DD) that are actually implemented and tested (components are encircled in green, the integrations of the subsystems are also colored in green), as stated before in the following sections we will explain in details what exactly has been implemented.
\begin{figure}[H]
	\begin{center}
		\hspace*{-40pt}
		\includegraphics[scale=0.2]{GeneralArchitecture.png}
	\end{center}
\caption{General Architecture - Implemented subsystems}
\end{figure}

\newpage

Our software includes all the functionalities that:
\begin{itemize}
	\item Allows the users to have a calendar of events which can be either scheduled or non scheduled;
	\item Computes feasible travels that allows the users to reach their scheduled events in the allotted time;
	\item Allows the users to define their own preference profiles, that are to be applied to the travels proposed by our system;
	\item Allows the users to save their preferred locations into their profiles;
	\item Allows the users to define flexible breaks and to define periodical events;
	\item Allows the users to save their tickets and associate them to compatible travel components.
\end{itemize}

In this first release are not included the functionalities that:
\begin{itemize}
	\item Allows the users to buy new tickets within our application;
	\item Allows the users to locate the nearest sharing vehicles;
	\item Allows the system to take into account weather forecast, strikes and traffic info and to notify the user if the path is no more feasible due to these issues.
\end{itemize} 
We've decided to not include them in our first release since we consider them as nice-to-have feature but not mandatory for an initial release, since they depend on the functionalities we've implemented and so they must came after those functionalities and since we have had to choose how to allocate properly the limited amount of time given for the implementation phase and these functionalities would have required more time to be implemented.
Here we provide a list of use cases, taken from our RASD document (section 3.2.3) that are satisfied in this first release:
\begin{itemize}
	\item [UC1] Registration;
	\item [UC2] Login;
	\item [UC3] View calendar;
	\item [UC4] Create event;
	\item [UC5] Create personalized event profiles;
	\item [UC6] Define flexible breaks;
	\item [UC7] Arrange trips - Only with ticket inserted by the user;
	\item [UC9] Add ticket possessed;
	\item [UC10] Obtain feasible travel paths - Without the alternatives paths;
	\item [UC11] Choose between overlapping events;
	\item [UC12] Edit event - Only in Application server, not yet available on Android app; 
	\item [UC13] Delete event;
	\item [UC14] Edit personalized event profiles;
	\item [UC15] Delete personalized event profiles.
\end{itemize}

\section{Application Server and Database Server}
\label{sec:ApplAndDBServers}
In this section we specify what requirements are actually implemented in our ApplicationServer subsystem; we will also examine every sub-component and state which requirements that are related to them are implemented in this first release of Travlendar+, and we also state the motivations for including them and excluding others. (see also chapter 5 of DD - requirement traceability).
\begin{figure}[H]
	\begin{center}
		\hspace*{-60pt}
		\includegraphics[scale=0.4]{ApplicationServer.png}
	\end{center}
\caption{The implemented components of the Application Server}
\end{figure}

\subsection{Authentication Manager}
All the requirements related to the authentication manager are fulfilled in this first release, the user can register himself into the system, log in from any device he likes, modify his account info, delete his account and request a new password. As specified in the previous documents, these are the requirement met by this manager:
\begin{itemize}
	\item \textbf{[R1]} the system checks if the e-mail inserted is real;
	\item \textbf{[R2]} a user cannot sign up with the same e-mail twice;
	\item \textbf{[R3]} the e-mail and password inserted must be correct;
	\item \textbf{[R4]} incorrect credentials prevent the user from logging in;
	\item \textbf{[R38]} a user must be logged.
\end{itemize}

\subsection{Event Manager}
Event manager handle the insertion, deletion and modification of both events and break events into the user's calendar.
It interacts with both path and schedule manager in order to provide feasible paths to the events and to decide whether they can be scheduled or not.
In the application server is also handled properly the propagation of periodical events through the time.
In our system is not yet handled the possibility to define events without an ending time since we do not consider this feature essential but just an optional and not so necessary feature. 
As specified in the previous documents, these are the requirement met by this manager:
\begin{itemize}
	\item \textbf{[R5]} a user must specify all mandatory fields to add the new event;
	\item \textbf{[R6]} the system reserves the specified time-slot for the event;
	\item \textbf{[R7]} the system warns the user if the inserted event overlaps with an already existing one;
	\item \textbf{[R29]} the system allows the user to specify a flexible interval and a minimum amount of time to schedule a break.
\end{itemize}
And these are the requirement not yet implemented in this manager:
\begin{itemize}
	\item \textbf{[R8]} if the ending time of an event is not specified, the systems considers as ending time the hour of departure for the successive event;
	\item \textbf{[R9]} when an event is inserted after an event without a specified ending time, the ending time  of the first event is anticipated as stated in [R8].
\end{itemize}

\subsection{Preference Manager}
All the requirements related to the preference manager are fulfilled.
This module offers all the functionalities needed to insert,modify and delete the user's event profiles and the user's preferred locations and is also able to apply the user's preferences to the feasible travels computed by our system.
As specified in the previous documents, these are the requirement met by this manager:
\begin{itemize}
	\item \textbf{[R12]} the system does not consider paths that violate constraints on travel means defined by the user;
	\item \textbf{[R13]} the system checks user preferences to decide which feasible travel path is the best;
	\item \textbf{[R15]} appropriate travel means must be suggested according to the type of event that they are related to;
	\item  \textbf{[R23]} the system requires minimum and maximum length allowed for a path to impose a constraint on a travel mean;
	\item \textbf{[R24]} the system requires an interval of time allowed to impose a constraint on a travel mean;
	\item \textbf{[R25]} the system does not consider solutions that violate constraints;
	\item \textbf{[R26]} the system allows the user to specify one or more travel means that cannot be used;
	\item \textbf{[R27]} the system does not consider solutions that include deactivated travel means.
\end{itemize}

\subsection{Path Manager}
This module manage the feasible travels computation, after the insertion or the modification of an event. It interacts with the \textit{PreferenceManager} in order to obtain the information needed to guarantee that the user preferences are respected in every proposed path.
The functionalities that allows to change a selected path, choosing between a set of feasible alternatives, and to obtain also options that include sharing vehicles are not yet available; we've chosen to not include them due to time-related issued and since they would be nice-to-have features but not essential for an initial release.
As specified in the previous documents, these are the requirement met by this manager:
\begin{itemize}
	\item \textbf{[R10]} every travel path proposed must be feasible in the available time (the interval between two consecutive events);
	\item \textbf{[R11]} if the travel involves two or more travel means, the starting location of the first proposed travel path and the ending location of the last proposed travel path must coincide respectively with the starting location and the ending location of the whole planned travel;
	\item \textbf{[R12]} the system does not consider paths that violate constraints on travel means defined by the user;
	\item \textbf{[R13]} the system checks user preferences to decide which feasible travel path is the best;
	\item \textbf{[R15]} appropriate travel means must be suggested according to the type of event that they are related to;
	\item \textbf{[R25]} the system does not consider solutions that violate constraints;
	\item \textbf{[R27]} the system does not consider solutions that include deactivated travel means;
	\item \textbf{[R35]} the system provides information about time of departure and arrival of the proposed travels.
\end{itemize}
And these are the requirement not yet implemented in this manager:
\begin{itemize}
	\item \textbf{[R18]} the system must show to the user all possibilities to reach a location in according with the requirements of [G4];
	\item \textbf{[R19]} alternative feasible travel paths must not generate overlappings with other events of the schedule;
	\item \textbf{[R28]} for each travel path, the system estimates its carbon footprint produced;
	\item \textbf{[R37]} the system provides information about travel time with shared vehicles.
\end{itemize}

\subsection{Schedule Manager}
This module is able to check if an event overlaps with other events and to guarantee that the flexible breaks are respected. It also checks that the user's travels do not overlap with other events. When an event overlaps with another one, it puts in a separate list of non scheduled events and it manage the user's requests of rescheduling these events.
As specified in the previous sections events without an ending time and alternative travels are not yet implemented into our system since, due to time-related issues, we've decided to exclude them from our first release of Travlendar+.
As specified in the previous documents, these are the requirement met by this manager:
\begin{itemize}
	\item \textbf{[R6]} the system reserves the specified time-slot for the event;
	\item \textbf{[R7]} the system warns the user if the inserted event overlaps with an already existing one;
	\item \textbf{[R14]} the system warns the user if it is not possible to arrive at an event location before its starting time;
	\item \textbf{[R20]} the combination of the travel paths proposed for the day must be feasible in the allotted time;
	\item \textbf{[R21]} if there are multiple events at the same time the system will propose in the schedule only the first event added;
	\item \textbf{[R22]} if the user forces into the schedule an event that overlaps with events already present in the schedule, these are removed from the schedule;
	\item \textbf{[R30]} if there is enough time for a break, the system reserves it within the specified flexible interval;
	\item \textbf{[R31]} if there is not enough time into the flexible interval specified, a warning is thrown.
\end{itemize}

And these are the requirement not yet implemented in this manager:
\begin{itemize}
	\item \textbf{[R8]} if the ending time of an event is not specified, the systems considers as ending time the hour of departure for the successive event;
	\item \textbf{[R9]} when an event is inserted after an event without a specified ending time, the ending time  of the first event is anticipated as stated in [R8];
	\item \textbf{[R19]} alternative feasible travel paths must not generate overlappings with other events of the schedule.
\end{itemize}

\subsection{Trip Manager}
Trip manager is the last module we've implemented and offers only the functionalities that allows the user to save and visualize his tickets and to associate them to travel components for which they are applicable.
This module does not yet provide functionalities that allows the user to buy public transportation tickets and to locate the nearest vehicles since they would have required the integration of other external APIs and we have decided to allocate our time in others functionalities that we consider more important than these ones.
As specified in the previous documents, these are the requirement met by this manager:
\begin{itemize}
	\item \textbf{[R32]} the system allows the user to specify all the ticket he already owns;
	\item \textbf{[R33]} the system shows to the user if he already holds a ticket for a proposed travel;
\end{itemize}
And these are the requirement not yet implemented in this manager:
\begin{itemize}
	\item \textbf{[R34]} the system allows the user to buy public transportation tickets according to proposed travels;
	\item \textbf{[R36]} the system shows to the user where sharing vehicles are located.
\end{itemize}

\subsection{Complication manager and Notification Manager}
Both Complication and Notification manager are not implemented in this first release of Travlendar+, since we've considered their functionalities not essential for the  first release of our system and since we had to make a choice due to limited implementation time. 
\section{Android App}
\label{sec:AndroidApp}
\todo{TODO}
