\subsection{MySQL}
\label{subsec:Middleware}
MySQL is a very popular and open source DBMS, it is easy to install and configure and we used it to manage the server database. MySQL doesn't present particular disadvantages.\\
We have linked MySQL with our Application Server by means of JPA, and we haven't encountered relevant difficulties while performing this integration.

\subsection{SQLite and Room}
\label{subsec:SQLite and Room}
SQLite is a relational DBMS widely used as embedded database software for local/client storage in application software. It is ACID-compliant and implements most of the SQL standard, using a dynamically and weakly typed SQL syntax. \\\\
In this implementation we also used the Room persistence library, that provides an abstraction layer over SQLite to allow fluent database access while harnessing the full power of SQLite. \\\\
There are three major components in Room:
\begin{itemize}
\item \textit{Database}: Contains the database holder and serves as the main access point for the underlying connection to the relational data;
\item \textit{Entity}: Represents a table within the database;
\item \textit{Data Access Object (DAO)}: Contains the methods used to access the database.
\end{itemize}

\subsection{GlassFish}
\label{subsec:GlassFish}
GlassFish is the server used to run our server-side application, it is open source and allows the execution of Java EE functions. By means of EJB APIs, moreover, it is possible to perform security functions. Unfortunately, we encountered some difficulties related to GlassFish:
\begin{itemize}
\item we have installed GlassFish 5.0 on a virtual server with a low amount of RAM: 1GB. GlassFish 5.0 require 1GB of RAM (but 2GB are recommended) to run, so it often goes down due to our virtual server hardware specifications. We solved this problem increasing the virtual server RAM with a swap in secondary memory;
\item GF 5.0 is relatively recent and contains some bugs that will be fixed in the following updates. For instance, we found that GF 5.0 is incompatible with JodaTime library.
\end{itemize}