We used \textit{Junit} in order to test single classes of our project that don’t contain DB dependencies or injected elements. With Junit we can test single functionalities, both in general and limit cases, and find errors in a quite short time. \\\\
\textit{Arquillan} is used to test entities classes…… \todo{TODO} \\\\
\textit{Mockito} allows to “mock” (simulate) functionalities of components that are yet tested or that will be tested in a different moment. We use Mockito to test the PathManager, a Stateless EJB that requires other EJBs to perform its tasks. It is not the standard situation in which the test framework is usually used; for this reason it was really difficult to find a way to configure properly the test environment.\\
All the functionalities performed by one of the injected beans were mocked: the idea is to take them as working and to check the instructions defined in the class tested. The strength of mock functions is very useful if you want to test particular instructions or specific execution cases on the code. 
