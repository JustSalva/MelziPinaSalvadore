In this chapter we explain how our source code is structured. \\
In our GitHub repository we have put both Application Server and Android app source code into two different folders contained into the Implementation folder: \href{https://github.com/JustSalva/MelziPinaSalvadore/tree/master/Implementation}{\color{blue}link}. \\
Since they are structured as separate projects we will explain their internal structure in two different sections:
\section{Android app}
\begin{figure}[H]
	\dirtree{%
		.1 /.
		.2 app.
		.3 src.
		.4 main.
		.5 java/it/polimi/travlendarplus \DTcomment{main Android app's package}.
		.6 activity
\DTcomment{contains activities related to layouts}.
		.7 fragment
\DTcomment{contains fragments classes}.
		.7 handler
\DTcomment{contains classes that handle server responses}.
		.8 event
\DTcomment{handlers related to event responses}.
		.8 location
\DTcomment{handlers related to location responses}.
		.8 preference
\DTcomment{handlers related to preference responses}.
		.8 ticket
\DTcomment{handlers related to ticket responses}.
		.7 listener
\DTcomment{contains listeners that allow click and drag actions}.
		.7 tasks
\DTcomment{contains classes that write on the local DB}.
		.6 database
\DTcomment{contains components of the local DB used by Room}.
		.7 converters
\DTcomment{converters from boxed to primitive data}.
		.7 dao.
		.7 entity.
		.7 viewModel
\DTcomment{contains classes that store UI-related data}.
		.6 retrofit
\DTcomment{contains classes used to perform requests to the server}.
		.5 res
\DTcomment{contains all the resources}.
		.5 AndroidManifest.xml
\DTcomment{provides information about your app}.
		.4 test/java/it/polimi/travlendarplus.
		.3 build.gradle
\DTcomment{defines build configurations}.
	}
\end{figure}

\section{ApplicationServer}
\begin{figure}
	\dirtree{%
		.1 /.
		.2 src.
		.3 main.
		.4 java/it/polimi/travlendarplus \DTcomment{main ApplicationServer's package}.
		.5 RESTful \DTcomment{contains all REST interfaces}.
		.6 RESTfulCalendar \DTcomment{contains all calendar REST interfaces}.
		.6 authenticationManager \DTcomment{contains all authentication REST interfaces}.
		.6 messages \DTcomment{contains all exchanged messages with clients}.
		.7 authenticationMessages.
		.7 calendarMessages.
		.7 tripMessages.
		.5 beans \DTcomment{contains all the java beans source code}.
		.6 calendarManager \DTcomment{source code of calendar related Java beans}.
		.7 googleMapsUtilities \DTcomment{source code of GMaps requests}.
		.7 support \DTcomment{source code of path computation support classes}.
		.6 emailManager \DTcomment{source code that handle email forwarding}.
		.6 tripManager \DTcomment{source code of trip related Java beans}.
		.5 entities \DTcomment{source code of JPA entity classes}.
		.6 calendar.
		.6 preferences.
		.6 tickets.
		.6 travelMeans.
		.6 travels.
		.5 exceptions \DTcomment{custom exception classes}.
		.6 authenticationExceptions.
		.6 calendarExceptions.
		.6 encryptionExceptions.
		.6 googleMapsExceptions.
		.6 persistenceExceptions.
		.6 tripManagerExceptions.
		.4 resources.
		.5 META-INF.
		.6 beans.xml \DTcomment{configuration file of bean classes}.
		.6 persistence.xml \DTcomment{configuration file of JPA}.
		.3 test.
		.4 java/it/polimi/travlendarplus \DTcomment{main test package}.
		.5 beans \DTcomment{contains all the Java beans test code}.
		.6 calendarManager.
		.7 googleMapsUtilities.
		.7 pathManager.
		.7 preferenceManager.
		.7 scheduleManager.
		.7 support.
		.5 entities \DTcomment{contains all the JPA entities test code}.
		.6 calendar.
		.6 preferences.
		.6 tickets.
		.6 travelMeans.
		.6 travels.
		.4 resources-glassfish-embedded \DTcomment{contains glassfish-embedded arquillian profile config files}.
		.4 resources \DTcomment{contains arquillian config files}.
		.2 pom.xml \DTcomment{ contains necessary info to build the project}.
}
\end{figure}