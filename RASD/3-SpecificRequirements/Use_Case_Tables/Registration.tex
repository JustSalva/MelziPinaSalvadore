\begin{table}[H]
	\begin{center}
		\begin{tabular}{ | p{0.3\textwidth} | p{0.7\textwidth} | }
		\hline
		Participating actors & Generic visitor.\\
		\hline
		Entry Condition & There are no entry conditions.\\
		\hline
		Event Flow & 
			\begin{enumerate}
				\item The visitor clicks on the "“Register”" button displayed onto the homepage;
				\item The visitor fills all the mandatory fields shown, including his email, his password (twice), his name, his surname and a captcha;
				\item The visitor clicks on "“Confirm”" button;
				\item The visitor receives a confirmation email and clicks on the confirmation link;
				\item The system saves all user data inserted.
			\end{enumerate} \\
		\hline
		Exit Condition & The visitor's registration is completed successfully, so the visitor is registered as an user of Travlendar+ and he can log in the system as a registered user. \\
		\hline
		Exception & If:
				\begin{itemize}
   					\item The visitor inserts an email already connected to an existing account;
   					\item The visitor inserts invalid info in at least one of the mandatory fields;
   					\item The visitor leaves empty at least one of the mandatory fields;
   				\end{itemize}
   		Then the system will request the visitor to complete/revise all uncorrected field, highlighting them.
If the visitor does not activate the account, the activation link will expire after a month and all the user's data will be deleted.\\ 
		\hline
		\end{tabular}
	\end{center}
	\caption{Registration use-case}
\end{table}