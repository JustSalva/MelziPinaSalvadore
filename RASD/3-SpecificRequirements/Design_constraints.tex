\subsection{Standards compliance}
\label{subsect:Standards compliance}
	The system should produce a log containing all the action performed in the system, in order to able to perform a recovery in case of failures.
	\newline
	This recovery functionality is not implemented in this version of the system.
	\newline 
	Since the system memorize user's personal data it will have to respect privacy laws, in particular during the entire specification and development process we'll use as reference the General Data Protection Regulation (GDPR) (Regulation (EU) 2016/679), adopted in April 2016 by the European Union ( It becomes enforceable from 25 May 2018).
\subsection{Hardware limitations}
\label{subsect:Hardware limitations}
	In order to run on mobile devices, the application will need at least:
	\begin{itemize}
		\item 3G mobile internet connectivity;
		\item enough space in memory to install the application and store personal data;
		\item GPS module.
	\end{itemize}
	Since the application is primarily designed to be used on mobile phones while been away from home, low resolution and limited display size will be huge design constraints.
	\newline
	This also means that developing an effective user interface that can effectively show quickly and efficiently all the information necessary to the user will be a challenging task.
	\newline
	Also depending on the phone hardware, low processing power and small amount of memory are a matter of concern.

\subsection{Any other constraint}
\label{subsect:Any other constraint}
	Travlendar+ is dependent on external services to locate sharing vehicles, acquire tickets and produce travel paths. Because of this reason, it is critical to perform an update every time that one of these external functionalities changes, with the intent to avoid failures of the system.
	\newline
	Possible malfunctions of these external services should be accounted for during the development of Travlendar+, making sure that the system doesn't get hung up or crashes while trying to reach them.