\subsection{Definitions}
	\begin{itemize}
	\item Arrange trip: the system provides to the user all information about a travel. If the travel is related to one or more public travel means, the system helps the user to organize it. It is indicated if the user already holds a valid ticket for a certain travel or if he has to buy a required one. Purchase of a ticket is handled on the transport services' sites. Trip and travel are synonymous.
	\item Best path: it is the preferred travel option proposed to reach the event location among the feasible paths, according to a parameter specified by the user. Best path can be chosen according to one of these features: length, cost or environmental sustainability. Best path is the path taken into account and showed in the daily schedule; the user can substitute it with an alternative feasible path. 
	\item Break event: it is an optional event whose starting and ending time are flexible. The user can define this kind of event when he wants to reserve a certain amount of time in the schedule and he doesn't need to specify a starting time. For instance, a user could be able to specify that lunch must be possible every day between 11:30-
2:30, and it must be at least half an hour long, but the specific timing is flexible. The app would then be sure to reserve, if possible, at least 30 minutes for lunch each day.
	\item Constraint: it is a rule defined on travel means by the user. When the system calculates feasible paths each constraint must be taken into account: travel options opposed to constraints are ignored. A constraint can be associated to a type of event. There are different types of constraint: min/max distance allowed for a mean, interval of hour in which it is possible to take a mean and possibility to deactivate a mean. 
	\item Feasible path: a path that allows the user to reach a specified location before the event to attend starts and observes the constraints defined for that event.  
	\item Google maps API: a set of functions offered by Google Maps to decide  the path between two locations.
	\item Overlapping event: an event (or a part of it) that happen in the same time of another event, added previously in the schedule. Because the schedule must be feasible, an event is considered as "overlapping" also if only its related travel overlap an event present into the schedule.
	\item Periodicity: it is related to events that occur more than one time. It can be defined as weekly, monthly or in which days of week the event happen.
	\item Schedule: a daily plan with a set of events inserted by the user that allows him to travel and attend all the events. If an event overlap other events, it can not be inserted into the schedule. 
	\item Type of event: a set of rules (constraints) that is defined by the user and can be associated to a specific event several times.
	\item Transport service provider: a public or private company that controls and supplies the transport with a travel mean. 
	\item Travel: it is used to indicate the path that the user has to go across in order to reach a location. It can be made by several travel components.
	\item Travel component: each single path that can be done with a travel mean. It has a starting and ending time, a departure and arrival location and a length. The union of one or more travel components creates a travel.
	\end{itemize}