\subsection{Goals}
Travlendar+ is a service based on a mobile application and a web application.
\newline
\newline
The user simply has to create events and insert them into his schedule; the application will then organize travels in order to reach locations in the best way, correctly inserting them within the daily schedule and notifying eventual overlappings. \newline
The user can modify the schedule with any feasible combination of added events and can specify different types of preferences on travel means, so the system can plan the trip according to his personal needs.
\newline
\newline
The principal goal of Travlendar+ is to save time, both in the construction of the schedule and in the traveling between events.
The main functionality of the system is to control the user's daily flow of events, helping him optimizing his time and making sure that all his preferences and constraints are respected.
Integration with external transport means providers is also offered, allowing the user to buy tickets and locate travel means without exiting the app.
\newline
\newline

\underline{Visitor} should be able to:
\begin{description}
\item [[G1]] sign up into the system;
\end{description}

\noindent\underline{User} should be able to:
\begin{description}
\item[[G2]] log into the system;
\item[[G3]] create event specifying location, date, starting and ending time;
\item[[G4]] obtain the best path according to his preferences and the list of eventual alternative paths to reach a location;
\item[[G5]] change the selected path with an alternative path; 
\item[[G6]] obtain a daily schedule that allows to attend to every event in program;
\item[[G7]] apply constraints on travel means;
	\begin{description}
	\item[[G7.1]] related to the length of travel;
	\item[[G7.2]] related to the period of day;
	\end{description}
\item[[G8]] deactivate one or more travel means;
\item[[G9]] select combinations of transportation means that minimize carbon footprint;
\item[[G10]] reserve time for lunch or break events;
\item[[G11]] arrange trips;
	\begin{description}
	\item[[G11.1]] buy needed tickets;
	\item[[G11.2]] find available sharing vehicle.
	\end{description}
\end{description}
