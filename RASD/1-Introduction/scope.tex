Travlendar+ is a service based on a mobile application and a web application.
\newline
\newline
The user simply has to create and add events into his schedule; the application will then, in order to reach locations in the most efficient way, organize travels, correctly inserting them within the daily schedule and notifying the user about eventual overlappings. \newline
The user can modify the schedule in the way he prefers and can specify different types of preferences on travel means, so that the system can plan the trip according to his personal needs.
\newline
\newline
The principal goal of Travlendar+ is to save time, both in the construction of the schedule and in the traveling between events.
\newline
The main functionality of the system is to control the user's daily flow of events, helping him optimizing his time and making sure that all his preferences and constraints are respected.
\newline
Integration with external transport means providers is also offered, allowing the user to buy tickets and locate travel means without exiting the app.
\newline

\subsection{World and shared phenomena}
This is an overview of the world in which Travlendar+ is intended to exist.
\newline
The system aims to plan events and appointments of the user in a way that allows him to attend every single one of them. Therefore, the world is made of entities related to travel. 
\newline
\newline
Shared phenomena allow the user to achieve his goals and affect both user and system behaviors:
\begin{itemize}
	\item Trips arrangement and schedules creation are phenomena controlled by the machine and observed by the world;
	\item Constraints definition and events creation are phenomena controlled by the world and observed by the machine.
\end{itemize}
The machine encloses the set of functions used to perform the different tasks. These items are not visible to the world.
\newline
\begin{figure} [h]
\centering
\includegraphics[scale=0.6]{shared_phenomena.jpg}
\end{figure}
