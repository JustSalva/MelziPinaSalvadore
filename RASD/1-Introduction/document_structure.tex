This RASD is composed by six parts:
\begin{enumerate}
\item The first part of the document contains an introduction to the problem, where goals are identified and basic information is provided in order to better understand the rest of the document.
\item The second part of the document describes generally the system, identifying its boundaries and the actors involved in the life-cycle of the system. Boundaries are defined providing all the necessary assumptions: both the ones required in order to adapt to the user's specifications and the ones that will hold and from now on will be considered as true.
\item The third part of the document is composed by:
	\begin{itemize}
	\item functional and non functional specific requirements defined;
	\item a list of nine scenarios is provided: each of them describes a particular situation that the system might have to cope with;
	\item UML diagrams that model in detail the system behavior.
	\end{itemize}
\item The fourth part of the document contains the Alloy model of the system, including all the relevant 
details. It also provides a proof of consistency and an example of the world generated.
\item The fifth part of the document contains a report of the hours spent to write this document.
\item The sixth part of the document contains references to external documents used in this document.
\end{enumerate}
