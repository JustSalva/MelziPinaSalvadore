\begin{table}[H]
	\begin{center}
		\begin{tabular}{ | p{0.3\textwidth} | p{0.7\textwidth} | }
		\hline
		Test case ID & T.C.1\\
		\hline
		Feature tested & Login  \\
		\hline
		Functional requirements & 1.1  \\
		\hline
		Test Steps & 
			\begin{enumerate}
				\item load Travlendar's home page in a browser;
				\item insert the credentials of a registered user;
				\item click on login.
			\end{enumerate} \\
		\hline
		Expected Result & The user is correctly logged.\\
		\hline
		Actual Result & The user is correctly logged.\\ 
		\hline
		Test Status & \color{ForestGreen}Success.\\ 
		\hline
		Comments & \\ 
		\hline
		\end{tabular}
	\end{center}
\end{table}

\begin{table}[H]
	\begin{center}
		\begin{tabular}{ | p{0.3\textwidth} | p{0.7\textwidth} | }
		\hline
		Test case ID & T.C.2\\
		\hline
		Feature tested & Visualize Calendar  \\
		\hline
		Functional requirements & 1.3  \\
		\hline
		Test Steps & 
			\begin{enumerate}
				\item open the Travlendar's home page while logged;
				\item observe the user's calendar.
			\end{enumerate} \\
		\hline
		Expected Result & The calendar is shown correctly, according to RASD and DD documents.\\
		\hline
		Actual Result & The calendar is shown correctly, according to RASD and DD documents.\\ 
		\hline
		Test Status & \color{ForestGreen}Success.\\ 
		\hline
		Comments & Maybe it would be better to display also the selected travels.  \\ 
		\hline
		\end{tabular}
	\end{center}
\end{table}

\begin{table}[H]
	\begin{center}
		\begin{tabular}{ | p{0.3\textwidth} | p{0.7\textwidth} | }
		\hline
		Test case ID & T.C.3\\
		\hline
		Feature tested & Visualize event's details  \\
		\hline
		Functional requirements & 1.4  \\
		\hline
		Test Steps & 
			\begin{enumerate}
				\item open the Travlendar's home page while logged;
				\item click on "arrow" icon inside an event box.
			\end{enumerate} \\
		\hline
		Expected Result & Further details about the journey are displayed.\\
		\hline
		Actual Result & A new tab is opened to display a path using Google Maps's website but the path is not the selected one. The opened tab display all possible paths, computed by Gmaps.  \\ 
		\hline
		Test Status & \color{ForestGreen}Success.\\ 
		\hline
		Comments & We have considered the test passed but only because in the end some travel information are shown, including the requested one.\\ 
		\hline
		\end{tabular}
	\end{center}
\end{table}

\begin{table}[H]
	\begin{center}
		\begin{tabular}{ | p{0.3\textwidth} | p{0.7\textwidth} | }
		\hline
		Test case ID & T.C.4\\
		\hline
		Feature tested & Travel means supported: walking, biking (own, or shared), bus, train, tram, taxis, driving (own or shared)  \\
		\hline
		Functional requirements & 2.1   \\
		\hline
		Test Steps & 
			\begin{enumerate}
				\item open the Travlendar's home page while logged;
				\item click on "+" button and fill the required fields;
				\item click on "generate" button in order to obtain feasible paths;
				\item repeat 10 or more times in order to obtain options related to all possible travel means.
			\end{enumerate} \\
		\hline
		Expected Result & All supported travel means are proposed to the user.\\
		\hline
		Actual Result & Only results with only generic info (duration and traveled distance).  \\ 
		\hline
		Test Status & \color{Red}failed.\\ 
		\hline
		Comments & For all the created events we have never observed proposed path with means different from walking, driving and "transit" that is the raw type used by GMaps to identify all public travel means, no additional info are shown to the user (ex. actual proposed public travel means) \\ 
		\hline
		\end{tabular}
	\end{center}
\end{table}

\begin{table}[H]
	\begin{center}
		\begin{tabular}{ | p{0.3\textwidth} | p{0.7\textwidth} | }
		\hline
		Test case ID & T.C.5\\
		\hline
		Feature tested & Deselect transportation options  \\
		\hline
		Functional requirements & 2.2   \\
		\hline
		Test Steps & 
			\begin{enumerate}
				\item open the Travlendar's home page while logged;
				\item click on setting's button;
				\item click on "available" check box of some means in order to deselect a travel mean;
				\item add an event and observe that deselected travel means are not proposed.
			\end{enumerate} \\
		\hline
		Expected Result & Deselected travel means are not shown as possible paths.\\
		\hline
		Actual Result & No paths filtering is performed, the user's preferences are basically ignored.\\ 
		\hline
		Test Status & \color{Red}failed.\\ 
		\hline
		Comments &\\ 
		\hline
		\end{tabular}
	\end{center}
\end{table}

\begin{table}[H]
	\begin{center}
		\begin{tabular}{ | p{0.3\textwidth} | p{0.7\textwidth} | }
		\hline
		Test case ID & T.C.6\\
		\hline
		Feature tested & Maximum walking
distance\\
		\hline
		Functional requirements & 2.3  \\
		\hline
		Test Steps & 
			\begin{enumerate}
				\item open the Travlendar's home page while logged;
				\item click on setting's button;
				\item in "Max walking distance" insert a max value;
				\item add an event and observe that the max value is respected.
			\end{enumerate} \\
		\hline
		Expected Result & Max walking distance is respected in the proposed feasible paths.\\
		\hline
		Actual Result & No paths filtering is performed, the user's preferences are basically ignored.\\ 
		\hline
		Test Status & \color{Red}failed.\\ 
		\hline
		Comments &\\ 
		\hline
		\end{tabular}
	\end{center}
\end{table}

\begin{table}[H]
	\begin{center}
		\begin{tabular}{ | p{0.3\textwidth} | p{0.7\textwidth} | }
		\hline
		Test case ID & T.C.7\\
		\hline
		Feature tested & Impose constraint that limits the usage of a specific public travel mean in a defined period of the day\\
		\hline
		Functional requirements & 2.4  \\
		\hline
		Test Steps & 
			\begin{enumerate}
				\item open the Travlendar's home page while logged;
				\item click on setting's button;
				\item set a time slot value and select the relative travel mean as available (ex. on "car");
				\item add an event and observe that preference is respected.
			\end{enumerate} \\
		\hline
		Expected Result & If the requested paths are in the period of the day in which the user does not want to use them the specified travel mean, they not include it.\\
		\hline
		Actual Result & No paths filtering is performed, the user's preferences are basically ignored.\\ 
		\hline
		Test Status & \color{Red}failed.\\ 
		\hline
		Comments &\\ 
		\hline
		\end{tabular}
	\end{center}
\end{table}

\begin{table}[H]
	\begin{center}
		\begin{tabular}{ | p{0.3\textwidth} | p{0.7\textwidth} | }
		\hline
		Test case ID & T.C.8\\
		\hline
		Feature tested & Choice to avoid walking or biking if the forecast is rain\\
		\hline
		Functional requirements & 2.6  \\
		\hline
		Test Steps & 
			\begin{enumerate}
				\item open the Travlendar's home page while logged;
				\item click on setting's button;
				\item set the "Dryness" value as "high";
				\item  add an event in a rainy day observe that preference is respected.
			\end{enumerate} \\
		\hline
		Expected Result & Walking and biking are ignored in rainy days.\\
		\hline
		Actual Result & Only a "rainy level" under a "drop" icon is displayed.\\ 
		\hline
		Test Status & \color{Red}failed.\\ 
		\hline
		Comments & We looked into the project's code and we've seen that an arbitrary value is set in this field \\ 
		\hline
		\end{tabular}
	\end{center}
\end{table}

\begin{table}[H]
	\begin{center}
		\begin{tabular}{ | p{0.3\textwidth} | p{0.7\textwidth} | }
		\hline
		Test case ID & T.C.9\\
		\hline
		Feature tested & Obtain environmentally-friendly travels\\
		\hline
		Functional requirements & 2.7 and 3.1\\
		\hline
		Test Steps & 
			\begin{enumerate}
				\item click on "+" button and fill the required fields;
				\item click on "generate" button in order to obtain feasible paths.
			\end{enumerate} \\
		\hline
		Expected Result & Environmentally-friendly travels are signaled.\\
		\hline
		Actual Result & Only a "environmental value" is displayed, we do not understand what it means.\\ 
		\hline
		Test Status & \color{Red}failed.\\ 
		\hline
		Comments & We looked into the project's code and we've seen that a value is computed, but we have observed that different values for different travels are incoherent form one another \\ 
		\hline
		\end{tabular}
	\end{center}
\end{table}

\begin{table}[H]
	\begin{center}
		\begin{tabular}{ | p{0.3\textwidth} | p{0.7\textwidth} | }
		\hline
		Test case ID & T.C.10\\
		\hline
		Feature tested & Prioritize a favorite travel mean\\
		\hline
		Functional requirements & 2.8\\
		\hline
		Test Steps & ???\\
		\hline
		Expected Result & Favorite travel means are handled in some way.\\
		\hline
		Actual Result & We have not observed such a functionality.\\ 
		\hline
		Test Status & \color{Red}failed.\\ 
		\hline
		Comments & We've seen that a relevance value is assigned to the proposed travels but we have not found where we should specify our preferred travel means. \\ 
		\hline
		\end{tabular}
	\end{center}
\end{table}

\begin{table}[H]
	\begin{center}
		\begin{tabular}{ | p{0.3\textwidth} | p{0.7\textwidth} | }
		\hline
		Test case ID & T.C.11\\
		\hline
		Feature tested & Specify a preferred location\\
		\hline
		Functional requirements & 2.9\\
		\hline
		Test Steps & 
			\begin{enumerate}
				\item open the Travlendar's home page while logged;
				\item click on setting's button;
				\item specify in the "home address" field a specific location
				\item ??? the specified location can not be selected while an user creates an event.
			\end{enumerate} \\
		\hline
		Expected Result & While creating an event there is an option that allows to select the "home location".\\
		\hline
		Actual Result & We have not observed such a functionality.\\ 
		\hline
		Test Status & \color{Red}failed.\\ 
		\hline
		Comments &\\ 
		\hline
		\end{tabular}
	\end{center}
\end{table}

--------------------------------------------------------------------

\begin{table}[H]
	\begin{center}
		\begin{tabular}{ | p{0.3\textwidth} | p{0.7\textwidth} | }
		\hline
		Test case ID & id\\
		\hline
		Feature tested & Breaks creation.\\
    	\hline
		Functional requirements & 5  \\
		\hline
		Test Steps & 
			\begin{enumerate}
				\item go to the settings page;
				\item click the add break button;
				\item fill the empty fields regarding the break information;
				\item ???.
			\end{enumerate} \\
		\hline
		Expected Result & A break event is inserted.\\
		\hline
		Actual Result & A break event is not inserted.\\ 
		\hline
		Test Status & \color{Red}Failed.\\ 
		\hline
		Comments & The button that should be used to save a break does not work. By looking at the code we found out that the function called on the button's onclick does not exist.\\
		\hline
		
		\end{tabular}
	\end{center}
\end{table}

\begin{table}[H]
	\begin{center}
		\begin{tabular}{ | p{0.3\textwidth} | p{0.7\textwidth} | }
		\hline
		Test case ID & id\\
		\hline
		Feature tested & Events collision.\\
    	\hline
		Functional requirements & 6.1  \\
		\hline
		Test Steps & 
			\begin{enumerate}
				\item go to the add event page;
				\item add an event;
				\item go to the add event page;
				\item try to add another event that takes place in the same time slot of the previously inserted one;
				\item a warning shows up saying that there is a collision detected;
				\item the event is not saved.
			\end{enumerate} \\
		\hline
		Expected Result & A warning shows up, the event is not saved.\\
		\hline
		Actual Result & A warning shows up, the event is not saved.\\ 
		\hline
		Test Status & \color{ForestGreen}Success.\\ 
		\hline
		Comments & If no break event is inserted, various other warnings show up, unrelated to the expected one. \\
		\hline
		
		\end{tabular}
	\end{center}
\end{table}

\begin{table}[H]
	\begin{center}
		\begin{tabular}{ | p{0.3\textwidth} | p{0.7\textwidth} | }
		\hline
		Test case ID & id\\
		\hline
		Feature tested & Event collision with a break.\\
    	\hline
		Functional requirements & 6.2  \\
		\hline
		Test Steps & 
			\begin{enumerate}
				\item add a break event;
				\item go to the add event page;
				\item try to add another event that takes place in the same time slot of the break event previously inserted;
				\item a warning shows up saying that there is a collision detected;
				\item the event is not saved.
			\end{enumerate} \\
		\hline
		Expected Result & A warning shows up, the event is not saved.\\
		\hline
		Actual Result & A warning shows up, the event is not saved.\\ 
		\hline
		Test Status & \color{ForestGreen}Success.\\ 
		\hline
		Comments & It is not possible to add a break event via the web application, we did that manually via PhpMyAdmin. The break event collision does not keep track of the event travel, e.g. it is possible to schedule a break during an event travel. \\
		\hline
		
		\end{tabular}
	\end{center}
\end{table}

\begin{table}[H]
	\begin{center}
		\begin{tabular}{ | p{0.3\textwidth} | p{0.7\textwidth} | }
		\hline
		Test case ID & id\\
		\hline
		Feature tested & Changing an appointment.\\
    	\hline
		Functional requirements & 6.3  \\
		\hline
		Test Steps & 
			\begin{enumerate}
				\item go to the calendar page;
				\item click on the pencil image of an event;
				\item edit the information of an event;
				\item click the save button;
				\item the event is saved.
			\end{enumerate} \\
		\hline
		Expected Result & The information edited about an event are saved.\\
		\hline
		Actual Result & The information edited about an event are saved.\\ 
		\hline
		Test Status & \color{ForestGreen}Success.\\ 
		\hline
		Comments & There are the same issues present also when trying to add an event.
		\end{tabular}
	\end{center}
\end{table}

\begin{table}[H]
	\begin{center}
		\begin{tabular}{ | p{0.3\textwidth} | p{0.7\textwidth} | }
		\hline
		Test case ID & id\\
		\hline
		Feature tested & Shortest travels calculation.\\
    	\hline
		Functional requirements & 7.1, 7.2  \\
		\hline
		Test Steps & 
			\begin{enumerate}
			\item go to the settings page;
			\item insert and save your preferences regarding travel means;
				\item go to the add event page;
				\item fill the required fields;
				\item click on 'generate';
				\item select the travel option wanted;
				\item save the event.
			\end{enumerate} \\
		\hline
		Expected Result & The application calculates the shortest travels and shows the alternatives available.\\
		\hline
		Actual Result & The application calculates the shortest travels, but the alternatives do not respect settings previously saved.\\ 
		\hline
		Test Status & \color{ForestGreen}Success.\\ 
		\hline
		Comments & Settings previously saved are not considered in the showing of possible travels.\\ 
		\hline
		
		\end{tabular}
	\end{center}
\end{table}

\begin{table}[H]
	\begin{center}
		\begin{tabular}{ | p{0.3\textwidth} | p{0.7\textwidth} | }
		\hline
		Test case ID & id\\
		\hline
		Feature tested & Purchasing of tickets.\\
    	\hline
		Functional requirements & 8.1  \\
		\hline
		Test Steps & 
			\begin{enumerate}
				\item go to the add event page;
				\item fill the required fields;
				\item click on 'generate';
				\item if there is a 'Transit travel' related to bus, train or tram, a link to the ATM website is showed;
				\item if that link is clicked, the user gets redirected (in the same tab) to the ATM website;
			\end{enumerate} \\
		\hline
		Expected Result & The user can purchase tickets.\\
		\hline
		Actual Result & The user can purchase tickets.\\ 
		\hline
		Test Status & \color{ForestGreen}Success.\\ 
		\hline
		Comments & It is not clear when the link to the ATM shows up, often it does not. \\ 
		\hline
		
		\end{tabular}
	\end{center}
\end{table}

\begin{table}[H]
	\begin{center}
		\begin{tabular}{ | p{0.3\textwidth} | p{0.7\textwidth} | }
		\hline
		Test case ID & id\\
		\hline
		Feature tested & Additions of passes.\\
    	\hline
		Functional requirements & 8.2  \\
		\hline
		Test Steps & 
			\begin{enumerate}
				\item ???;
			\end{enumerate} \\
		\hline
		Expected Result & The user can add information about his passes.\\
		\hline
		Actual Result & It is not possible to add information about passes.\\ 
		\hline
		Test Status & \color{Red}Failed.\\ 
		\hline
		Comments & It is not clear where this option is implemented. \\ 
		\hline
		
		\end{tabular}
	\end{center}
\end{table}

\begin{table}[H]
	\begin{center}
		\begin{tabular}{ | p{0.3\textwidth} | p{0.7\textwidth} | }
		\hline
		Test case ID & id\\
		\hline
		Feature tested & Information about the nearest shared vehicle.\\
    	\hline
		Functional requirements & 8.3  \\
		\hline
		Test Steps & 
			\begin{enumerate}
				\item ???;
			\end{enumerate} \\
		\hline
		Expected Result & The user can add find the nearest shared vehicle.\\
		\hline
		Actual Result & It is not possible to find the nearest shared vehicle.\\ 
		\hline
		Test Status & \color{Red}Failed.\\ 
		\hline
		Comments & It is not clear where this option is implemented. The option about adding a sharing vehicle doesn't even show up when a user is trying to add a new event. \\
		\hline
		
		\end{tabular}
	\end{center}
\end{table}

\begin{table}[H]
	\begin{center}
		\begin{tabular}{ | p{0.3\textwidth} | p{0.7\textwidth} | }
		\hline
		Test case ID & id\\
		\hline
		Feature tested & Warning message showed when destination is unreachable.\\
    	\hline
		Functional requirements & 9.1  \\
		\hline
		Test Steps & 
			\begin{enumerate}
				\item go to the add event page;
				\item fill the required fields;
				\item click on 'Generate';
			\end{enumerate} \\
		\hline
		Expected Result & A warning message is shown.\\
		\hline
		Actual Result & No warning message is shown.\\ 
		\hline
		Test Status & \color{Red}Failed.\\ 
		\hline
		Comments & The locations selection is unclear, often when inserting popular locations like Como, the wrong city gets selected. This leads to no travels showing up. It is not clear what is the difference between an unreachable destination and a not found one. It is impossible to insert an unreachable destination given that the user settings do not impact on the creation of a travel. \\
		\hline
		
		\end{tabular}
	\end{center}
\end{table}

\begin{table}[H]
	\begin{center}
		\begin{tabular}{ | p{0.3\textwidth} | p{0.7\textwidth} | }
		\hline
		Test case ID & id\\
		\hline
		Feature tested & Profile creation.\\
    	\hline
		Functional requirements & 10.1  \\
		\hline
		Test Steps & 
			\begin{enumerate}
				\item go to the login page;
				\item click on 'Create a user';
				\item fill the required fields;
				\item click on create.
			\end{enumerate} \\
		\hline
		Expected Result & The user can create a profile.\\
		\hline
		Actual Result & The user can create a profile.\\ 
		\hline
		Test Status & \color{ForestGreen}Success.\\ 
		\hline
		Comments & What is inserted in the registration fields is not controlled. For instance, it is possible to leave the fields empty and create a new profile with username, password and email empty. \\
		\hline
		
		\end{tabular}
	\end{center}
\end{table}