\begin{table}[H]
	\begin{center}
		\begin{tabular}{ | p{0.3\textwidth} | p{0.7\textwidth} | }
		\hline
		Test case ID & id\\
		\hline
		Feature tested & Breaks creation.\\
    	\hline
		Functional requirements & 5  \\
		\hline
		Test Steps & 
			\begin{enumerate}
				\item go to the settings page;
				\item click the add break button;
				\item fill the empty fields regarding the break information;
				\item ???.
			\end{enumerate} \\
		\hline
		Expected Result & A break event is inserted.\\
		\hline
		Actual Result & A break event is not inserted.\\ 
		\hline
		Test Status & \color{Red}Failed.\\ 
		\hline
		Comments & The button that should be used to save a break does not work. By looking at the code we found out that the function called on the button's onclick does not exist.\\
		\hline
		
		\end{tabular}
	\end{center}
\end{table}

\begin{table}[H]
	\begin{center}
		\begin{tabular}{ | p{0.3\textwidth} | p{0.7\textwidth} | }
		\hline
		Test case ID & id\\
		\hline
		Feature tested & Events collision.\\
    	\hline
		Functional requirements & 6.1  \\
		\hline
		Test Steps & 
			\begin{enumerate}
				\item go to the add event page;
				\item add an event;
				\item go to the add event page;
				\item try to add another event that takes place in the same time slot of the previously inserted one;
				\item a warning shows up saying that there is a collision detected;
				\item the event is not saved.
			\end{enumerate} \\
		\hline
		Expected Result & A warning shows up, the event is not saved.\\
		\hline
		Actual Result & A warning shows up, the event is not saved.\\ 
		\hline
		Test Status & \color{ForestGreen}Success.\\ 
		\hline
		Comments & If no break event is inserted, various other warnings show up, unrelated to the expected one. \\
		\hline
		
		\end{tabular}
	\end{center}
\end{table}

\begin{table}[H]
	\begin{center}
		\begin{tabular}{ | p{0.3\textwidth} | p{0.7\textwidth} | }
		\hline
		Test case ID & id\\
		\hline
		Feature tested & Event collision with a break.\\
    	\hline
		Functional requirements & 6.2  \\
		\hline
		Test Steps & 
			\begin{enumerate}
				\item add a break event;
				\item go to the add event page;
				\item try to add another event that takes place in the same time slot of the break event previously inserted;
				\item a warning shows up saying that there is a collision detected;
				\item the event is not saved.
			\end{enumerate} \\
		\hline
		Expected Result & A warning shows up, the event is not saved.\\
		\hline
		Actual Result & A warning shows up, the event is not saved.\\ 
		\hline
		Test Status & \color{ForestGreen}Success.\\ 
		\hline
		Comments & It is not possible to add a break event via the web application, we did that manually via PhpMyAdmin. The break event collision does not keep track of the event travel, e.g. it is possible to schedule a break during an event travel. \\
		\hline
		
		\end{tabular}
	\end{center}
\end{table}

\begin{table}[H]
	\begin{center}
		\begin{tabular}{ | p{0.3\textwidth} | p{0.7\textwidth} | }
		\hline
		Test case ID & id\\
		\hline
		Feature tested & Changing an appointment.\\
    	\hline
		Functional requirements & 6.3  \\
		\hline
		Test Steps & 
			\begin{enumerate}
				\item go to the calendar page;
				\item click on the pencil image of an event;
				\item edit the information of an event;
				\item click the save button;
				\item the event is saved.
			\end{enumerate} \\
		\hline
		Expected Result & The information edited about an event are saved.\\
		\hline
		Actual Result & The information edited about an event are saved.\\ 
		\hline
		Test Status & \color{ForestGreen}Success.\\ 
		\hline
		Comments & There are the same issues present also when trying to add an event.
		\end{tabular}
	\end{center}
\end{table}

\begin{table}[H]
	\begin{center}
		\begin{tabular}{ | p{0.3\textwidth} | p{0.7\textwidth} | }
		\hline
		Test case ID & id\\
		\hline
		Feature tested & Shortest travels calculation.\\
    	\hline
		Functional requirements & 7.1, 7.2  \\
		\hline
		Test Steps & 
			\begin{enumerate}
			\item go to the settings page;
			\item insert and save your preferences regarding travel means;
				\item go to the add event page;
				\item fill the required fields;
				\item click on 'generate';
				\item select the travel option wanted;
				\item save the event.
			\end{enumerate} \\
		\hline
		Expected Result & The application calculates the shortest travels and shows the alternatives available.\\
		\hline
		Actual Result & The application calculates the shortest travels, but the alternatives do not respect settings previously saved.\\ 
		\hline
		Test Status & \color{ForestGreen}Success.\\ 
		\hline
		Comments & Settings previously saved are not considered in the showing of possible travels.\\ 
		\hline
		
		\end{tabular}
	\end{center}
\end{table}

\begin{table}[H]
	\begin{center}
		\begin{tabular}{ | p{0.3\textwidth} | p{0.7\textwidth} | }
		\hline
		Test case ID & id\\
		\hline
		Feature tested & Purchasing of tickets.\\
    	\hline
		Functional requirements & 8.1  \\
		\hline
		Test Steps & 
			\begin{enumerate}
				\item go to the add event page;
				\item fill the required fields;
				\item click on 'generate';
				\item if there is a 'Transit travel' related to bus, train or tram, a link to the ATM website is showed;
				\item if that link is clicked, the user gets redirected (in the same tab) to the ATM website;
			\end{enumerate} \\
		\hline
		Expected Result & The user can purchase tickets.\\
		\hline
		Actual Result & The user can purchase tickets.\\ 
		\hline
		Test Status & \color{ForestGreen}Success.\\ 
		\hline
		Comments & It is not clear when the link to the ATM shows up, often it does not. \\ 
		\hline
		
		\end{tabular}
	\end{center}
\end{table}

\begin{table}[H]
	\begin{center}
		\begin{tabular}{ | p{0.3\textwidth} | p{0.7\textwidth} | }
		\hline
		Test case ID & id\\
		\hline
		Feature tested & Additions of passes.\\
    	\hline
		Functional requirements & 8.2  \\
		\hline
		Test Steps & 
			\begin{enumerate}
				\item ???;
			\end{enumerate} \\
		\hline
		Expected Result & The user can add information about his passes.\\
		\hline
		Actual Result & It is not possible to add information about passes.\\ 
		\hline
		Test Status & \color{Red}Failed.\\ 
		\hline
		Comments & It is not clear where this option is implemented. \\ 
		\hline
		
		\end{tabular}
	\end{center}
\end{table}

\begin{table}[H]
	\begin{center}
		\begin{tabular}{ | p{0.3\textwidth} | p{0.7\textwidth} | }
		\hline
		Test case ID & id\\
		\hline
		Feature tested & Information about the nearest shared vehicle.\\
    	\hline
		Functional requirements & 8.3  \\
		\hline
		Test Steps & 
			\begin{enumerate}
				\item ???;
			\end{enumerate} \\
		\hline
		Expected Result & The user can add find the nearest shared vehicle.\\
		\hline
		Actual Result & It is not possible to find the nearest shared vehicle.\\ 
		\hline
		Test Status & \color{Red}Failed.\\ 
		\hline
		Comments & It is not clear where this option is implemented. The option about adding a sharing vehicle doesn't even show up when a user is trying to add a new event. \\
		\hline
		
		\end{tabular}
	\end{center}
\end{table}

\begin{table}[H]
	\begin{center}
		\begin{tabular}{ | p{0.3\textwidth} | p{0.7\textwidth} | }
		\hline
		Test case ID & id\\
		\hline
		Feature tested & Warning message showed when destination is unreachable.\\
    	\hline
		Functional requirements & 9.1  \\
		\hline
		Test Steps & 
			\begin{enumerate}
				\item go to the add event page;
				\item fill the required fields;
				\item click on 'Generate';
			\end{enumerate} \\
		\hline
		Expected Result & A warning message is shown.\\
		\hline
		Actual Result & No warning message is shown.\\ 
		\hline
		Test Status & \color{Red}Failed.\\ 
		\hline
		Comments & The locations selection is unclear, often when inserting popular locations like Como, the wrong city gets selected. This leads to no travels showing up. It is not clear what is the difference between an unreachable destination and a not found one. It is impossible to insert an unreachable destination given that the user settings do not impact on the creation of a travel. \\
		\hline
		
		\end{tabular}
	\end{center}
\end{table}

\begin{table}[H]
	\begin{center}
		\begin{tabular}{ | p{0.3\textwidth} | p{0.7\textwidth} | }
		\hline
		Test case ID & id\\
		\hline
		Feature tested & Profile creation.\\
    	\hline
		Functional requirements & 10.1  \\
		\hline
		Test Steps & 
			\begin{enumerate}
				\item go to the login page;
				\item click on 'Create a user';
				\item fill the required fields;
				\item click on create.
			\end{enumerate} \\
		\hline
		Expected Result & The user can create a profile.\\
		\hline
		Actual Result & The user can create a profile.\\ 
		\hline
		Test Status & \color{ForestGreen}Success.\\ 
		\hline
		Comments & What is inserted in the registration fields is not controlled. For instance, it is possible to leave the fields empty and create a new profile with username, password and email empty. \\
		\hline
		
		\end{tabular}
	\end{center}
\end{table}